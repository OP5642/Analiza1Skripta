\documentclass[../main.tex]{subfiles}

\usepackage[OT2]{fontenc}
\usepackage[english,serbianc]{babel} 
\usepackage{listings}%dozvoljava nam da dodajemo delove koda
\usepackage{amsmath}%dozvoljava nam neke stvari iz matematike
\usepackage{amsthm}%theorem stvari
\usepackage{dsfont}%npr. za boldovanu jedinicu
\usepackage{amsfonts}%dozvoljava upotrebu raznih fontova
\usepackage{mathrsfs}%za neke fontove, npr mathscr za filtere
\usepackage{mathtools} %svashta, npr. za pisanje teksta iznad \iff
\usepackage{amssymb}%dozvoljava upotrebu raznih simbola puput measuredangle
\usepackage{color} %koristimo boje
\usepackage{enumitem,framed} %za leftmargin u itemiz
\usepackage{indentfirst}
\usepackage[lmargin=2.0cm, rmargin=2.0cm,tmargin=2.50cm,bmargin=2.50cm]{geometry}

\usepackage{subfiles} %najbolje ovo na kraju koristiti

\definecolor{mygreen}{rgb}{0,0.5,0}
\definecolor{mygray}{rgb}{0.7,0.7,0.7}
\definecolor{mymauve}{rgb}{0.58,0,0.82}
 
\righthyphenmin 2 


%Podesavanje izgleda koda
\lstset{ 
 backgroundcolor=\color{mygray}, % choose the background color
 basicstyle=\footnotesize, % the size of the fonts that are used for the code
 breakatwhitespace=false, % sets if automatic breaks should only happen at whitespace
 breaklines=true, % sets automatic line breaking
 captionpos=b, % sets the caption-position to bottom
 commentstyle=\color{blue}, % comment style
 deletekeywords={...}, % if you want to delete keywords from the given language
 escapeinside={\%*}{*)}, % if you want to add LaTeX within your code
 extendedchars=true, % lets you use non-ASCII characters; for 8-bits encodings only, does not work with UTF-8
 frame=single, % adds a frame around the code
 keepspaces=true, % keeps spaces in text, useful for keeping indentation of code (possibly needs columns=flexible)
 keywordstyle=\color{mygreen}, % keyword style
 language=TeX, %defining the prefered language, can be changned in []
 morekeywords={*,...}, % if you want to add more keywords to the set
 numbers=left, % where to put the line-numbers; possible values are (none, left, right)
 numbersep=5pt, % how far the line-numbers are from the cod
 numberstyle=\tiny\color{mygray}, % the style that is used for the line-numbers
 rulecolor=\color{black}, % if not set, the frame-color may be changed on line-breaks within not-black text (e.g. comments (green here))
 showspaces=false, % show spaces everywhere adding particular underscores; it overrides 'showstringspaces'
 showstringspaces=false, % underline spaces within strings only
 showtabs=false, % show tabs within strings adding particular underscores
 stepnumber=1, % the step between two line-numbers. If it's 1, each line will be numbered
 stringstyle=\color{mymauve}, % string literal style
 tabsize=2, % sets default tabsize to 2 spaces
 title=\lstname % show the filename of files included with \lstinputlisting; also try caption instead of title
}
%Kraj podesavanja izgleda koda
\begin{document}
\section{Primitivna funkcija}

{\de $F$ je primitivna za $f$ na $I$ ($F$ je reshenje diferencijalne jednachine $F'=f$ na $I$), gde je $I \psj \R$ pravi interval $(\sup I > \inf I)$ i $\K \in\{\R, \C\}, f:I\to \K, F:I\to \K $ ako vaz1i:
 \begin{itemize}
        \item[(1)] $F$ je neprekidna na $I$
        \item[(2)] postoji najvishe prebrojiv skup $P\subseteq I$ takav da je: $(\forall x \in I\setminus P) F'(x)=f(x)$.
\end{itemize} 
$F$ je tachna primitivna za $f$ na $I$ ako vaz1i:
\begin{itemize}
        \item[(1)] $F$ je neprekidna na $I$
        \item[(2)] $(\forall x \in I) F'(x)=f(x)$.
\end{itemize}
}
{\pr Neka je $f:\mathbb{R}\to \mathbb{R}, f(x)= \sign x$ i $F:\mathbb{R}\to \mathbb{R}, F(x)= |x|$. $F$ je neprekidna na $\mathbb{R}$ i vaz1i $F'(x)=f(x)$ za $x \in \mathbb{R}\setminus \{0\}$ stoga sledi da je $F$ primitivna za $f$ na $\mathbb{R}$}.
{\pr Neka je $f$ karakteristichna funkcija intervala $I=[a,b]$, odnosno funkcija $f:\mathbb{R}\to \mathbb{R}$ i $f(x)=\chi_{[a,b]}(x)$, to jest: $$
f(x)=\begin{cases}
			1, & x\in I\\
            0, & x\in \mathbb{R}\setminus I
		 \end{cases}$$
i neka je $F$ funkcija zadata na sledec1i nachin:
$$F(x)=\begin{cases}
			0, & x<a\\
            x-a, & a\mj x \mj b\\
            b-a, & x>b
		 \end{cases}$$
Kako je $F$ neprekidna na $\mathbb{R}$ i $F'(x)=f(x)$ za $x\in\mathbb{R}\setminus\{a,b\}$ zakljuchujemo da je $F$ primitivna za $f$ na $\mathbb{R}$.}
{\pr Neka su $f$ i $F$ funckije zadate na sledec1i nachin:
$$F:\mathbb{R}\to \mathbb{R},\hspace{1cm} F(x)=\begin{cases}
			x^2 \sin \frac{1}{x}, & x\ne 0\\
            0, & x=0
		 \end{cases}$$
$$f:\mathbb{R}\to \mathbb{R},\hspace{1cm} f(x)=\begin{cases}
			2x \sin \frac{1}{x} - \cos \frac{1}{x}, & x\ne 0\\
            0, & x=0
		 \end{cases}$$
Funkcija $F$ je neprekidna na $\mathbb{R}$ jer je neprekidna u nuli $\ds{(\lim_{x\to 0^{+}}F(x)=\lim_{x\to 0^{-}}F(x)}=F(0) )$, a u tachkama razlichitim od nule neprekidna je kao kompozicija neprekidnih. Primetimo da je $F$ tachna primitivna za $f$ jer uvek vaz1i $F'(x)=f(x)$}.
{\pr Neka je $I\psj \mathbb{R}$ otvoren interval i $F:I\to\mathbb{R}$ konveksna funkcija. Odatle zakljuchujemo da $F\in \mathscr{D}_{+}(I) \cap \mathscr{D}_{-}(I)$ i $F'_{+}, F'_{-}$ rastu, pa sledi da $F'_{+}$ i $F'_{-}$ imaju najvishe prebrojivo mnogo prekida i svi oni su prekidi prve vrste (ranije dokazano kod monotonosti). Kako iz 
$$F'_{-}C_{+}a \implies (\forall x \in I) (x>a \implies F'_{-}(a)\mj F'_{+}(a) \mj F'_{-}(x)),$$
prelaskom na $\ds{\lim_{x\to a^{+}}}$ dobijamo $F'_{+}(a) = F'_{-}(a)$, shto znachi da je $F \mathscr{D} a$ u svakoj tachki $a\in I$ u kojoj je $F'_{-}C_{+}a$, odnosno svuda osim u prebrojivo mnogo tachaka. Definishimo skup $P$ na sledec1i nachin: $P=\{a\in I \hspace{0.3cm} | \hspace{0.3cm} \lnot F'_{-}C_{+}a\} $ i 
$$f(x)=\begin{cases}
			F'(x), & x\in I\setminus P\\
            \textrm{bilo shta}, & x\in P,
		 \end{cases}$$
tada je $F$ primitivna za $f$. Stoga, svaka konveksna funkcija je primitivna.}\\ \\
\nap Navodimo i opshte tvrd1enje uz prethodne primere.Ako je:

\begin{itemize}
        \item[$\cdot$] $I\psj \R$ pravi interval $(\sup I > \inf I)$
        \item[$\cdot$] $P\psj I$ najvishe prebrojiv podskup
        \item[$\cdot$] $F:I\to \C , F\in C(I) \cup \mathscr{D}(I\setminus P),$
\end{itemize}
onda je $F$ primitivna funkcija za $f:I\to \C$
$$f(x)=\begin{cases}
			F'(x), & x\in I\setminus P\\
            \textrm{bilo shta}, & x\in P
		 \end{cases}$$
\begin{tvr}
Funkcija $F:I\to \C$ je primitivna za $f:I \to \C$ ako i samo ako je $\Re F$ primitivna za $\Re f$ i $\Im F$ primitivna za $\Im f$.
\end{tvr}
\begin{tvr}
\textbf{(Jedinstvenost primitivne funkcije)} Neka je funkcija $F:I \to \C$ primitivna za $f:I \to \C$. Tada je funkcija $G: I\to \C$ primitivna za $f$ ako i samo ako je 
$$F-G\equiv const$$
\end{tvr}
\begin{proof}
Smer $\impliedby$ sledi iz $(\textrm{\latin{const}})'=0$ i neprekidnosti funkcije $F$ na intervalu $I$. Smer $\implies$ dokazujemo na sledec1i nachin. Iz uslova tvrd1enja znamo da je $F'=f$ na $I\razl P_1$ i $G'=f$ na $I\razl P_2$, gde su $P_1$ i $P_2$ najvishe prebrojivi, i $F, G \in C(I)$. Sledi da je $(F-G)'=F'-G'=f-f=0$ na $I\razl ( P_1 \cup P_2 ) $, shto jeste komplement najvishe prebrojivog skupa poshto je unija dva najvishe prebrojiva skupa najvishe prebrojiv skup. Na osnovu opshte teoreme srednje vrednosti zakljuchujemo da je $F-G= const$.
\end{proof}
\nap Predhodno tvrd1enje nam zapravo govori da ukoliko su $F_1$ i $F_2$ dve primi-tivne funkcije funkcije $f$ na intervalu, tada je razlika $F_2(x)-F_1(x)$ konstantna. Vaz1no je obratiti paz1nju da su primitivne na intervalu. Zaista, neka su date funkcije 
$$ F_1(x)=x^2 \hspace{0.6cm} \text{i} \hspace{0.6cm} F_2(x)= \begin{cases}
			x^2, & x<0\\
            x^2+1, & x>0
		 \end{cases}$$
na skupu $(-\infty,0) \cup (0,+\infty)$ koji nije intreval. Tada je $F_1'(x)=F_2'(x)=2x$ za svako $x$ iz $(-\infty,0) \cup (0,+\infty)$, med1utim, $F_2(1)-F_1(1)=1\ne 0 = F_2(-1)-F_1(-1)$, te razlika $F_2-F_1$ nije konstantna.
\begin{posl}
Neka su $F, G:I \to \C$ primitivne za $f:I\to \C$. Tada vaz1i:
  \begin{itemize}
        \item[(1)] $(\exists x \in I)$  $F(x)=G(x) \iff (\forall x \in I)$  $F(x)=G(x)$
        \item[(2)] $F(x_1)-F(x_2) = G(x_1)-G(x_2)$ za sve $x_1, x_2 \in I$.
  \end{itemize} 
\end{posl}

\begin{posl}
Ako je funkcija $F:I \to \C$ primitivna za $f: I \to \C$, onda za svako $x_0 \in I$ i svako $y_0 \in \C$ postoji tachno jedna primitivna funkcija $\phi : I \to \C$ za $f$ takva da je $\phi (x_0) = y_0$. Pri tome je $\phi(x)=F(x)-F(x_0)+y_0$.
\end{posl}
\newpage
\subsection{Tablica primitivnih funkcija}
\begin{center}
\vspace{0.5cm}
\begin{tabular}{||c  c  c  c||} 
 \hline
 $f(x)$ & $F(x)$ & $\text{\latin {dom }}F$ & napomena \\ [0.5ex] 
 \hline\hline
 0 & $c$ & $\R$ & $c\in \C$ \\ [2ex] 
 %\hline
 $x^{\alpha}$ & $\frac{1}{\alpha +1} x^{\alpha +1}$ & zavisi od $\alpha$ & $\alpha \in \R , \alpha \ne -1$ \\[2ex] 
 %\hline
 $\frac{1}{x}$ & $\log|x|$ & $\R^{*}$ &  \\[2ex] 
 %\hline
 $\frac{1}{1+x^2}$ & $\arctg x$ & $\R$ & $\frac{\pi}{2} -\arctg x$ \\[2ex] 
 %\hline
  $\frac{1}{1-x^2}$ & $\frac{1}{2} \log |\frac{1+x}{1-x}|$ & $\R \razl \{-1,1\} $ &  \\ [2ex] 
 $\frac{1}{\sqrt{1-x^2}}$ & $\arcsin x$ & $[-1,1]$ & $\frac{\pi}{2}-\arccos x$ \\ [2ex] 
 %\hline
 $\frac{1}{\sqrt{1+x^2}}$ & $\log |x+\sqrt{x^2+1}|$ & $ \R $ & $\text{\latin {arcsh }} x$\\[2ex] 
 %\hline
 $\frac{1}{\sqrt{x^2-1}}$ & $\log |x+\sqrt{x^2-1}|$ & $\R\razl(-1,1)$ & $\text{\latin {arcch }} x, |x|\geqslant 1$ \\[2ex] 
 %\hline
 $e^{\xi x}$ & $\frac{1}{\xi} e^{\xi x}$ & $\R$ & $\xi \in \C^{*}$ \\[2ex] 
 %\hline
  $a^x$ & $\frac{1}{\log a} a^x $ & $\R $ & $a\in \R, a>0, a\ne 1$ \\ [2ex] 
  $\cos x$ & $\sin x$ & $\R$ &  \\ [2ex] 
 %\hline
 $\sin x$ & $ -\cos x$ & $\R$ &\\[2ex] 
 %\hline
 $\sec^2 x$ & $\tg x$ & $\R \razl \frac{\pi}{2}(2\Z+1)$ &  \\[2ex]
 %\hline
 $\cosec^2 x$ & $-\ctg x$ & $\R \razl \pi \Z$ &  \\[2ex] 
 %\hline
  $\ch x$ & $\sh x$ & $\R $ &  \\ [2ex] 
 %\hline
  $\sh x$ & $\ch x$ & $\R $ &  \\ [2ex] 
 \hline
\end{tabular}
\end{center}
\section {Neodred1eni integral}
\begin{de}
\textbf{Neodred1eni integral} funkcije $f:I \to \C$ je skup svih njenih
primitivnih funkcija na $I$. Oznaka: $\int f(x) dx$ ili $\int f$.
\end{de}
Ako je $F:I \to \C$ neka primitivna funkcija za $f$ na $I$, onda
je $$\int f(x) dx = \{F+C | c \in \C\}.$$
\begin{tvr}
(\textbf{Lokalnost primitivne funkcije}) Neka je $I\psj \R$ ogranichen interval.
Za funkcije $f:I\to \C$ i $F:I \to \C$ sledec1i uslovi su ekvivalentni:
\begin{itemize}
        \item[(a)] $F$ je primitivna za $f$ na $I$
        \item[(b)] $F$ je primitivna za $f$ na svakom kompaktnom
        intervalu $K\psj I$
        \item[(v)] Za svako $x \in I$ postoji otvoren interval $x \in \math{J_x}$ na kom je $F$ primitivna za $f$
\end{itemize} 
\end{tvr}
\begin{proof}
Kako je $P\psj I$ najvishe prebrojiv skup, zakljuchujemo da i $P\cap K$ takod1e najvishe prebrojiv. 
Iz toga i lokalnosti neprekidnosti i diferencijabilnosti zakljuchujemo da (a) implicira (b).\\
Dokaz1imo da (b) implicira (a). Neka je $K_n$ rastuc1i (u odnosu na inkluziju) niz kompaktnih intervala takav da je $I=\cup_{n\in \N} K_n$.
Takav niz uvek postoji, jer ako je $a_{\infty} = \inf I$ i $b_{\infty} = \sup I$,
onda postoje nizovi $a_n$ i $b_n$ u $I$ takvi da vaz1i:
\begin{itemize}
        \item[\cdot] $a_{\infty} = \ds{\lim_{n\to \infty}a_n}$,
        $b_{\infty} = \ds{\lim_{n\to \infty}b_n}$
        \item[\cdot] $a_{\infty} \in I \implies (\forall n \in \N) a_n=a_{\infty}, b_{\infty} \in I \implies (\forall n \in \N) b_n=b_{\infty}$
\end{itemize} 
Tada je $K_n:=[a_n,b_n]$ traz1eni niz. Iz (b) sledi da je za svako $n\in \N$ $F|_{K_n}$ primitivna za $f|_{K_n}$.
Odatle sledi da je za svako $n\in \N$ $F|_{K_n}$ neprekidna i za neki najvishe prebrojiv skup $P_n\psj K_n$
vaz1i $$x\in K_n \razl P_n \implies (F|_{K_n})'(x)=f|_{K_n}(x).$$ 
Kako je $I=\cup_{n\in \N} K_n$ zakljuchujemo da za $x \in I$ postoji prirodan broj n takav da $x \in K_n$.
Funkcija $F|_{K_n}$ je neprekidna kao primitivna, a odatle koristec1i se poznatim konceptom lokalnosti neprekidnosti zakljuchujemo da je $F\cont x$.
Oznachimo sa $P_{\infty}=\cup_{n\in \N}P_n$ najvishe prebrojiv skup. Tada ako $x\in I \razl P_{\infty}$ onda $x\in I \razl \cup_{n\in \N}P_n$.
Koristec1i De-Morganove zakone dobijamo $x\in \cap_{n\in \N}(I\razl P_n)$. Odatle sledi da $(F|_{K_n})'(x)=f|_{K_n}(x)$, odnosno $F'(x)=f(x)$, zbog lokalnosti diferencijabilnosti.\\
Trivijalno je da (a) implicira (v), uzmimo $\I_x = (\inf I, \sup I)$. Da iz (v) sledi (b) dokazujemo na sledec1i nachin. Neka je $K$ kompaktan interval i $\I_x$ kao u (v),
zakljuchujemo da $$(\forall x \in I)(\exists j \in \{1, \dots n\}) x\in \I_{x_j}.$$
Tada pod pretpostavkom (v) sledi da $(\forall j) F\cont \I_{x_j}$ i $F'=f$ na $\I_{x_j}\razl P_j$ za neki najvishe prebrojiv skup $P_j \psj \I_{x_j}$.
Zakljuchujemo da $F\cont I$ i $F'=f $ na $I\razl
\cup_{j=1}^n P_j$.
\end{proof}
\begin{tvr}
(\textbf{Linearnost integrala}) Neka je:
\begin{itemize}
        \item[\cdot] $F:I\to \C$ primitivna za $f:I\to \C$
        \item[\cdot] $G:I\to \C$ primitivna za $g:I\to \C$
        \item[\cdot] $\alpha, \beta \in \C$
\end{itemize}
Tada je $\alpha F + \beta G:I\to \C$ primitivna za $\alpha f + \beta g: I \to \C$.
\end{tvr}
\begin{proof}
Dokaz sledi iz linearnosti neprekidnosti, linearnosti izvoda i chinjenice da je unija dva najvishe prebrojiva skupa najvishe prebrojiv skup.
\end{proof}
\begin{posl}
Za $f,g:I \to \C$ i $\alpha, \beta \in \C$ je
$$\int (\alpha f(x) + \beta g(x)) dx = \alpha \int f(x) dx + \beta \int g(x) dx$$
Specijalno, $$\int f(x) dx = \int \Re f(x) dx + i\int \Im f(x)dx.$$
\end{posl}
\begin{pr}
$\int \tg^2 x dx = \int (\sec^2 x - 1) dx = \int \sec^2 x dx - \int dx = \tg x - x + C$
\end{pr}
\begin{pr}
$\int \frac{x^2}{1+x^2}dx= \int (1-\frac{1}{1+x^2})dx=\int dx - \int \frac{1}{1+x^2}dx= x-\arctg x +C$.
\end{pr}
\begin{pr}
$a,b \in \R$ 
\begin{eqnarray*}
 \int e^{ax} \cos{bx} dx+ i \int e^{ax} \sin{bx} dx&=& \int e^{(a+ib)x} dx\\
 &=& \frac{1}{a+ib} e^{(a+ib)x}+C \\
 &=& \frac{1}{a^2+b^2}(a-ib) e^{(a+ib)x}+C \\
 &=& \frac{e^{ax}}{a^2+b^2}(a-ib)(\cos bx + i \sin bx)+C
\end{eqnarray*}
Izjednachavanjem realnog dela sa realnim i imaginarnog sa imaginarnim dobijamo:
$$\int e^{ax} \sin bx dx=\frac{e^{ax}}{a^2+b^2}(a \sin bx - b \cos bx)+C$$
$$\int e^{ax} \cos bx dx=\frac{e^{ax}}{a^2+b^2}(a \cos bx+b \sin bx) +C$$
\end{pr}
\nap U poslednjem primeru smo koristili identitet:
$$e^{ax}\cdot e^{ibx}=e^{(a+ib)x}$$ 
Iz Ojlerovog identiteta
$$e^{ix}=\cos x+i \sin x=\text{\latin {cis}}\hspace{0.1cm} x$$
sledi $\frac{d}{dx} e^{ix}=i e^{ix}$. Neka je $f:\R \to \C$,
$f(x)=e^{ax}e^{ibx}-e^{(a+ib)x}$. Tada je $f'(x)=0$ pa je $f\equiv \const$. Kako je $f(0)=0$, sledi da je $f\equiv 0$.
\subsection{Osnovni metodi izrachunavanja neodred1enih integrala}
\begin{tvr}
(\textbf{Parcijalna integracija}) Neka su $F,G:I\to \C$ primitivne funkcije (shto znachi da su neprekidne na $I$
i diferencijabilne na komplementu najvishe prebrojivog skupa). Tada, ako jedna od funkcija $F'G$ i $FG'$ ima primitivnu, imac1e je i druga i vaz1ic1e
$$\int FG' = FG - \int F'G$$
\end{tvr}
\begin{proof}
Neka su $F,G$ primitivne, odnosno 
$F\in \cont(I) \cap \mathit D(I\razl P), G\in \cont(I) \cap \mathit D(I\razl T)$, gde su $P$ i $T$ najvishe prebrojivi skupovi.
Tada je i $P \cup T$ najvishe prebrojiv skup i vaz1i $\forall x \in I \razl (P\cup T) (FG)'(x)=F'(x)G(x)+F(x)G'(x)$. Odatle sledi da ukoliko $F'G$ ima primitivnu,
onda je ima i $FG'$ i vaz1i $\int FG' = FG - \int F'G$. 
\end{proof}
\begin{posl}
Ako $f:I\to \C$ ima primitivnu, onda je 
$$\int f(x) dx=x f(x) - \int x f'(x) dx.$$
\end{posl}

\begin{pr}
\begin{eqnarray*}
 \int \log x dx = x \log x - \int x (\log x)' dx 
 &=& x\log x - \int x \cdot \frac{1}{x} dx \\
 &=& x \log x - x + C
\end{eqnarray*}
\end{pr}

\begin{pr}
\begin{eqnarray*}
 \int \arcsin x dx = x \arcsin x - \int x (\arcsin x)' dx 
 &=& x\arcsin x - \int \frac{x}{\sqrt{1-x^2}}dx \\
 &=& x \arcsin x +\sqrt{1-x^2} + C
\end{eqnarray*}
\end{pr}

\begin{pr}
\begin{eqnarray*}
 \int \sin \log x dx &=& x \sin \log x - \int x (\sin \log x)' dx\\ 
 &=& x\sin \log x - \int x \cdot \cos \log x \cdot \frac{1}{x}dx \\
 &=& x\sin \log x - \int \cos \log x  dx \\
 &=& x\sin \log x - x \cos \log x + \int x \cdot (-\sin \log x) \cdot \frac{1}{x}dx \\
 &=& x(\sin \log x -  \cos \log x) - \int \sin \log x  dx \\
\end{eqnarray*}
Odatle dobijamo da je
$$2 \int \sin \log x = x(\sin \log x -  \cos \log x) + C$$
Odnosno
$$\int \sin \log x = \frac{x}{2}(\sin \log x - \cos \log x) + C$$
\end{pr}
\begin{tvr}
(\textbf{Uopshtena parcijalna integracija}) Neka su funkcije $F^{(n)}$
i $G^{(n)}$ primitivne. Ako jedna od funkcija $F^{(n+1)}G$ i $FG^{(n+1)}$
ima primitivnu, onda je ima i druga i vaz1i:
$$\int FG^{(n+1)} = \sum_{j=0}^{n} (-1)^j F^{(j)} G^{(n-j)} + (-1)^{n+1} \int F^{(n+1)} G.$$
\end{tvr}
\begin{proof}
Iz Lajbnicovog pravila sledi
\begin{eqnarray*}
 \left(\sum_{j=0}^n(-1)^jF^{(j)}G^{(n-j)}\right)' &=& \sum_{j=0}^n(-1)^j\left(F^{(j+1)}G^{(n-j)}+F^{(j)}G^{(n-j+1)}\right)\\ 
 &=& \sum_{j=0}^n(-1)^j F^{(j+1)}G^{(n-j)}+\sum_{j=0}^n(-1)^j F^{(j)}G^{(n-j+1)}\\
 &=& FG^{(n+1)}+(-1)^n F^{(n+1)}G \\
\end{eqnarray*}
$\implies$ Ako $F^{(n+1)}G$ ima primitivnu imac1e je i $FG^{(n+1)}$ i
vaz1ic1e
$$\int FG^{(n+1)} = \sum_{j=0}^{n} (-1)^j F^{(j)} G^{(n-j)} + (-1)^{n+1} \int F^{(n+1)} G.$$
\end{proof}
\begin{tvr}
(\textbf{Smena promenljive}) Neka vaz1i:
\begin{itemize}
        \item[(1)] $T\psj \R, X\psj \R$ su pravi intervali
        \item[(2)] $F:X \to \C$ je primitivna za $f:X\to \C$
        \item[(3)] $\varphi:T\to X$ je primitivna
        \item[(4)] $f$ je neprekidna ili je $\varphi$ strogo monotona
\end{itemize}
Tada je $F\circ \varphi:F\to \C$ primitivna za $(f\circ \varphi)\cdot \varphi':T\to \C$, tj. $\int(f\circ \varphi)\cdot \varphi'=F\circ \varphi +C$
\end{tvr}
Dato tvrd1enje moz1e se formulisati i na sledec1i nachin:
\begin{tvr}
Neka je na intervalu $(a,b)$ 
$$\int f(x) dx=F(x)+C$$
i neka je funkcija $\varphi:(\alpha, \beta)\to(a,b)$ neprekidno diferencijabilna (tj. diferencijabilna, sa neprekidnim prvim izvodom) i na. Tada je
$$\int(f\circ \varphi(t))\cdot \varphi'(t)=F\circ \varphi(t) +C
\hspace{0.3cm} (*)$$
\end{tvr}
Jednakost $(*)$ dobija se diferenciranjem obe strane i primenom pravila za izvod sloz1ene funkcije. Ona je u osnovi metoda izrachunavanja integrala smenom promenljive koji se sastoji u sledec1em. 
Ukoliko z1elimo da izrachunamo integral oblika $\int(f\circ \varphi(t))\cdot \varphi'(t)$ smenom $x=\varphi(t), dx=\varphi'(t) dt$
dobijamo integral koji moz1e da bude jednostavniji od polaznog.
\begin{pr}
(\textbf{Linearna smena promenljive}) Neka je $F:I\to \C$ primitivna
za $f:I\to \C$, $a\in \R^*, b \in \R$. Tada je $\frac{1}{a}F(ax+b)$
primitivna za $f(ax+b)$ na $\I=\frac{1}{a}(I-b)$ i
$$\int f(ax+b) dx=\frac{1}{a}F(ax+b)+C.$$
Funkcija $\varphi:\I \to I$ definisana kao $\varphi(x)=ax+b$ je strogo monotona i
diferencijabilna. Iz prethodnog tvrd1enja sledi da je 
$F\circ \varphi(x)=F(ax+b)$ primitivna funkcija za $f\circ \varphi \cdot \varphi'(x)=f(ax+b)\cdot a ,$ pa je $\frac{1}{a}F(ax+b)$ 
primitivna za $f(ax+b)$.
\end{pr}
\begin{pr}
Neka je $f:I\to \R$ neprekidna, $0\notin f(I), P\psj I$ najvishe prebrojiv i $f\in \it{D} (I\razl P)$. Tada je 
$$\int \frac{f'}{f}=\log|f|+C.$$
Neka je $g(x)=\log|x|$, $gC\R^*$, $f(I)\psj \R^*$. Odatle sledi da je
$\log|f|=g\circ f \in \cont(I)$. Iz diferencijabilnosti funkcije $g$
na skupu $\R^*$ i diferencijabilnosti funkcije $f$ na $I\razl P$ zakljuchujemo da je $g\circ f$ diferencijabilna na $I
\razl P$.
Kako za $x \in I\razl P$ vaz1i $(\log|f|)'=\frac{f'}{f}$ sledi da je $\log|f|$ primitivna za $\frac{f'}{f}$.
\end{pr}
\begin{pr}
$$\int \frac{1}{\sin x} dx = \int \frac{1}{2 \sin{\frac{x}{2}}\cos{\frac{x}{2}}}dx = \int \frac{1}{2} \cdot 
\frac{1}{\cos^2{\frac{x}{2}}} \cdot \frac{1}{\tg{\frac{x}{2}}} dx
= \int (\tg\frac{x}{2})'\frac{1}{\tg\frac{x}{2}} dx = \log|\tg\frac{x}{2}|+C$$

\end{pr}
\end{document}
