\documentclass[../main_og.tex]{subfiles}

\usepackage[OT2]{fontenc}
\usepackage[english,serbianc]{babel} 
\usepackage{listings}%dozvoljava nam da dodajemo delove koda
\usepackage{amsmath}%dozvoljava nam neke stvari iz matematike
\usepackage{amsthm}%theorem stvari
\usepackage{dsfont}%npr. za boldovanu jedinicu
\usepackage{amsfonts}%dozvoljava upotrebu raznih fontova
\usepackage{mathrsfs}%za neke fontove, npr mathscr za filtere
\usepackage{mathtools} %svashta, npr. za pisanje teksta iznad \iff
\usepackage{amssymb}%dozvoljava upotrebu raznih simbola puput measuredangle
\usepackage{color} %koristimo boje
\usepackage{enumitem,framed} %za leftmargin u itemiz
\usepackage{indentfirst}
\usepackage[lmargin=2.0cm, rmargin=2.0cm,tmargin=2.50cm,bmargin=2.50cm]{geometry}

\usepackage{subfiles} %najbolje ovo na kraju koristiti

\definecolor{mygreen}{rgb}{0,0.5,0}
\definecolor{mygray}{rgb}{0.7,0.7,0.7}
\definecolor{mymauve}{rgb}{0.58,0,0.82}
 
\righthyphenmin 2 


%Podesavanje izgleda koda
\lstset{ 
 backgroundcolor=\color{mygray}, % choose the background color
 basicstyle=\footnotesize, % the size of the fonts that are used for the code
 breakatwhitespace=false, % sets if automatic breaks should only happen at whitespace
 breaklines=true, % sets automatic line breaking
 captionpos=b, % sets the caption-position to bottom
 commentstyle=\color{blue}, % comment style
 deletekeywords={...}, % if you want to delete keywords from the given language
 escapeinside={\%*}{*)}, % if you want to add LaTeX within your code
 extendedchars=true, % lets you use non-ASCII characters; for 8-bits encodings only, does not work with UTF-8
 frame=single, % adds a frame around the code
 keepspaces=true, % keeps spaces in text, useful for keeping indentation of code (possibly needs columns=flexible)
 keywordstyle=\color{mygreen}, % keyword style
 language=TeX, %defining the prefered language, can be changned in []
 morekeywords={*,...}, % if you want to add more keywords to the set
 numbers=left, % where to put the line-numbers; possible values are (none, left, right)
 numbersep=5pt, % how far the line-numbers are from the cod
 numberstyle=\tiny\color{mygray}, % the style that is used for the line-numbers
 rulecolor=\color{black}, % if not set, the frame-color may be changed on line-breaks within not-black text (e.g. comments (green here))
 showspaces=false, % show spaces everywhere adding particular underscores; it overrides 'showstringspaces'
 showstringspaces=false, % underline spaces within strings only
 showtabs=false, % show tabs within strings adding particular underscores
 stepnumber=1, % the step between two line-numbers. If it's 1, each line will be numbered
 stringstyle=\color{mymauve}, % string literal style
 tabsize=2, % sets default tabsize to 2 spaces
 title=\lstname % show the filename of files included with \lstinputlisting; also try caption instead of title
}
%Kraj podesavanja izgleda koda

\begin{document}
    Pristupamo definisanju realnih brojeva preko aksiome supremuma, kao i algebarskim strukturama. Ovo nije jedini pristup, moguc1e su i opcije preko Arhimedove i Kantorove aksiome, koje c1emo pominjati dalje u toku poglavalja. Takodje je moguc1e i primeniti Dedekindove preseke za zasnivanje realnih brojeva, svi od ovih nachina su ekvivalentni, shto nec1emo eksplicitno dokazati, ali ovde nagalshavamo.

\section{Abelove i (totalno) uredjene Abelove grupe}

    {\de Algebarska struktura $(A,+,0)$ je \textit{Abelova grupa} ako su ispunjeni sledec1i uslovi:

\begin{itemize}[itemindent=6em]
    \item[(Zatvorenost)] $(\forall x,y\in A) x+y\in A$  
    \item[(Asocijativnost)] $(\forall x,y,z \in A) x+(y+z)=(x+y)+z$ 
    \item[(Neutral)] $(\exists 0 \in A) (\forall x \in A) x+0=x=x+0$ 
    \item[(Inverz)] $(\forall x\in A) (\exists -x\in A) x+(-x)=(-x)+x=0$  
    \item[(Komutativnost)]  $(\forall x,y\in A)x+y=y+x$  
\end{itemize}}

    Zatvorenost je zapravo svojstvo same operacije, pa se ponekad ne navodi u samoj defi-niciji. Struktura koja je samo zatvorena naziva se gruopoid, ako je uz to josh i asocija-tivna, onda je semigrupa (pulugrupa), ako josh i postoji neutral onda je monoid, a ako postoji i inverz onda je grupa. Komutativne grupa je Abelova grupa. Ova strukutra prirodno opisuje neke osnovne stvari, npr. $(\mathbb{Z},+,0)$ je primer Abelove grupe, $(S_n,\circ,)$ je grupa permutacija (nije komutativna u opshtem sluchaju), i u njima vazhe stvari koje su prirodne za ochekivati, npr. jedinstvenost neutrala i inverza.    

    {\tvr Neutral je jedinstven.}

    \begin{proof} 
        Pretpostavimo suprotno, tj. da imamo dva razlichita neutrala, $0_1$ i $0_2$. Iz defini-cije neutrala imamo $0_1=0_1+0_2=0_2$, pa je neutral jedinstven.\end{proof}

    Prirodno je definisati i oduzimanje (slichnu ideju koristimo i kada formalno, u logici, definishemo cele brojeve), naime $z$ je razlika $x$ i $y$, u oznaci $z=x-y$, ako je $x+z=y$. Formalnije, definishimo funkciju $\tau_a : A \to A$ (ovakvu funkciju nazivamo translacijom iz ochiglednih razloga), gde je $\tau_a(x)=a+x$. Posebno istichemo sledec1a svojstva.
    {\tvr Za svaku translaciju vazhi:
        \begin{itemize}
    \item $\tau_{a+b}=\tau_a\circ\tau_b=\tau_b\circ\tau_a$
\item $\tau_a\circ\tau_{-a}=\tau_{-a}\circ\tau_a=\mathds{1}_a$
\item $\tau_{-a}=\tau_a^{-1}$
\item Translacija je permutacija skupa $A$.\end{itemize}}
    \begin{proof}
        Prvo svojstvo direktno sledi iz asocijativnosti operacije $+$ i komutativnosti. Ono direktno povalchi drugo, iz kojeg zakljuchujemo da su $\tau_a$ i $\tau_{-a}$ medjusobno inverzne bijekcije, pa odatle zakljuchujemo i poslednje dve stavke.\end{proof}

    {\tvr Za sve $x,y,z\in A$ jednachina $\tau_y(z)=x$ ima tachno jedno reshenje $z$ }
    \begin{proof}
    Sledi iz trec1e stavke prethodnog tvrdjenja.     \end{proof}

    {\posl Inverz je jedinstven (inverz je u stvari reshenje jednachine $\tau_x(y)=0$).} 

    Sada i mozhemo formalno rec1i da je $z$ razlika brojeva $x$ i $y$, u prethodno navedenoj oznaci, ako je $\tau_{y}(z)=x$

    {\de Preslikavanje $\sigma: A \to A$ dato sa $\sigma(x)=-x$ nazivamo simetrijom, ili refleksijom}.

    \nap Refleksija je dobro definisana jer je inverz jedinstven.

    {\tvr Za svaku refleksiju vazhi $\sigma\circ\sigma=\mathds{1}$}

    \begin{proof} 
    Iz definicije sledi da je $x+\sigma(x)=0$, a samim tim je i $\sigma(x)+\sigma(\sigma(x))=0$, odakle sledi da je $\sigma(\sigma(x))=x$, za proizvoljno $x$.    \end{proof}

    Sabiranje se definishe kao funkcija $+:A \times A \to A$. Medjutim, mi smo navikli da pishemo i npr. izraze oblika $a+b+c+d$, koji bi imali smisla samo kada imamo vec1 asocijativnost operacije. Definiciju sabiranja proshirujemo na preslikavanje $\Sigma : \cup_{n\in \mathbb{N}} A^{n} \to A$ na induktivan nachin.
\end{document}
