\documentclass[../main.tex]{subfiles}

\usepackage[OT2]{fontenc}
\usepackage[english,serbianc]{babel} 
\usepackage{listings}%dozvoljava nam da dodajemo delove koda
\usepackage{amsmath}%dozvoljava nam neke stvari iz matematike
\usepackage{amsthm}%theorem stvari
\usepackage{dsfont}%npr. za boldovanu jedinicu
\usepackage{amsfonts}%dozvoljava upotrebu raznih fontova
\usepackage{mathrsfs}%za neke fontove, npr mathscr za filtere
\usepackage{mathtools} %svashta, npr. za pisanje teksta iznad \iff
\usepackage{amssymb}%dozvoljava upotrebu raznih simbola puput measuredangle
\usepackage{color} %koristimo boje
\usepackage{enumitem,framed} %za leftmargin u itemiz
\usepackage{indentfirst}
\usepackage[lmargin=2.0cm, rmargin=2.0cm,tmargin=2.50cm,bmargin=2.50cm]{geometry}

\usepackage{subfiles} %najbolje ovo na kraju koristiti

\definecolor{mygreen}{rgb}{0,0.5,0}
\definecolor{mygray}{rgb}{0.7,0.7,0.7}
\definecolor{mymauve}{rgb}{0.58,0,0.82}
 
\righthyphenmin 2 


%Podesavanje izgleda koda
\lstset{ 
 backgroundcolor=\color{mygray}, % choose the background color
 basicstyle=\footnotesize, % the size of the fonts that are used for the code
 breakatwhitespace=false, % sets if automatic breaks should only happen at whitespace
 breaklines=true, % sets automatic line breaking
 captionpos=b, % sets the caption-position to bottom
 commentstyle=\color{blue}, % comment style
 deletekeywords={...}, % if you want to delete keywords from the given language
 escapeinside={\%*}{*)}, % if you want to add LaTeX within your code
 extendedchars=true, % lets you use non-ASCII characters; for 8-bits encodings only, does not work with UTF-8
 frame=single, % adds a frame around the code
 keepspaces=true, % keeps spaces in text, useful for keeping indentation of code (possibly needs columns=flexible)
 keywordstyle=\color{mygreen}, % keyword style
 language=TeX, %defining the prefered language, can be changned in []
 morekeywords={*,...}, % if you want to add more keywords to the set
 numbers=left, % where to put the line-numbers; possible values are (none, left, right)
 numbersep=5pt, % how far the line-numbers are from the cod
 numberstyle=\tiny\color{mygray}, % the style that is used for the line-numbers
 rulecolor=\color{black}, % if not set, the frame-color may be changed on line-breaks within not-black text (e.g. comments (green here))
 showspaces=false, % show spaces everywhere adding particular underscores; it overrides 'showstringspaces'
 showstringspaces=false, % underline spaces within strings only
 showtabs=false, % show tabs within strings adding particular underscores
 stepnumber=1, % the step between two line-numbers. If it's 1, each line will be numbered
 stringstyle=\color{mymauve}, % string literal style
 tabsize=2, % sets default tabsize to 2 spaces
 title=\lstname % show the filename of files included with \lstinputlisting; also try caption instead of title
}
%Kraj podesavanja izgleda koda
\begin{document}

\section{Rimanov integral}

    Cilj je definisati (eventualno i izrachunati) povrshinu ispod grafika zadate funk-cije $f:[a,b]\to \R$ (bitno je da je funkcija na intervalu). 
    Ovo je problem koji su i Stari Grci, Arhimed i Eudoks, umeli da reshe za prilichno
    shiroke klase funkcija.

    Moguc1e je pristpupiti problemu sa vishe gledishta, ili preko posmatranja limesa, ili preko supremuma i infimuma. Prvi nachin je Rimanov, a drugi Darbuov i oba ima-ju
    svoje mane, kao i prednosti. Kada dokazhemo njihovu ekvivalenciju, moc1i c1emo da upotrebljavamo oba u zavisnosti od toga koji nam vishe odgovara u datoj situaciji.

\subsection{Veza sa primitivnom funkcijom}

    Pretpostavimo da je funkcija $f$ neprekidna na intervalu $[a,b]$ i oznachimo sa $P(x)$ povrshinu ispod grafika funkcije $f$ na intervalu $[a,x]$.
    Dakle, ako imamo $h \to 0$, onda je i $f(x+h)\to f(x)$, odakle sledi $P(x+h)-P(x) \sim f(x) h$ (tj. razliku povrshina posmatramo zapravo kao povrshinu pravougaonika).
    Odavde imamo da je \[f(x) \sim \dfrac{P(x+h)-P(x)}{h}\xrightarrow{h\to 0}P'(x)\\
	\implies P'(x)=f(x).
    \]
    Dakle, ako je $f\cont [a,b]$, $P$ je primitivna za $f$.

    Naglasimo da je ovo neformalan dokaz, sluzhi nam chisto radi intuicije. Ovo tvrdjenje c1emo kasnije nazivati osnovnom teoremom integralnog rachuna.
    Ovo nam daje intuitivnu predstavu o sledec1oj definiciji.

    \begin{de}
    	Neka je $F:[a,b]\to \C$ primitivna funkcija neprekidne funkcije $f:[a,b]\to \C$. Njutnov integral funkcije $f$ je 
	\[ \int_a^bf(x)dx=F(b)-F(a).\]
    \end{de}
    Iz jedinstvenosti primtivne funkcije $F$ za $f$ na intervalu (do na aditivnu konstan-tu) sledi da je definicija dobra, tj. da Njutnov integral ne zavisi od odabira
    primitivne funkcije.

    Razlika $F(b)-F(a)$ se oznachava i sa $F|_a^b$, $F(x)|_a^b$, $F(x)|_{x=a}^{x=b}$, a umesto $\int_a^bf(x)dx=F(x)|_a^b$ pishemo $\int_a^bf=F|_b^a$.
    
    No, prirodno je postaviti pitanje kada uopshte postoji primitvna funkcija? Odavde je jasno da za neprekidne postoji primitivna funkcija, ali to je veoma
    jaka pretpostvka. Iz datih definicija i primera ne mozhemo da zakljuchimo kada postoji primitivna funkcija date funkcije. Odogovor c1e nam dati formalna definicija
    Rimanovog integrala.

    Mozhemo formulisati i opsthiji problem. Naime, zashto se zausaviti na povrshini? Chesto c1e nas zanimati i zapremina pa chak i vishe dimenziona zapremina. No pre toga
    treba definisati pojam analogan intervalu u vishe dimenzija.

    \begin{de}
    	Neka su $\mcl{I}_1,\mcl{I}_2,\dotso,\mcl{I}_n$ intervali u $\R$. Dekartov proizvod $\mcl{J}=\mcl{I}_1 \times \mcl{I}_2 \times \dotso \times \mcl{I}_n$ nazivamo 
	$n$-dimenzionim	kvadrom.
    \end{de}

    Krajnji cilj je definisati (ukoliko je moguc1e i izrachunati) $(n+1)$-dimenzionu zapreminu ispod grafika funkcije $f:\mcl{J} \to \R$.

    U sluchaju $n=1$ imamo \zn 2-dimenzionu zapreminu\zng, shto je zapravo povrshina, a u sluchaju $n=2$ imamo \zn 3-dimenzionu zapreminu\zng, shto je zapravo 
    standardna zapremina.

\subsection{Kostrukcija Rimanovog integrala}

    \begin{de}
    	Podela intervala $[a,b]$ je konachan skup tachaka $\mcl{P}=\{x_0,x_1,\dotso,x_k\}$, takvih da je $a=x_0<x_1<x_2<\dotso<x_k=b$. Svaku tachku koja je chlan podele
	intervala nazivamo podeonom tachkom, a svaki od intervala $[x_i,x_{i+1}],i\in\{0,1,\dotso,n-1\}$ nazivamo podeonim intervalom.
    \end{de}

    Neka su $\mcl{P}_1$ i $\mcl{P}_2$ podele intervala $[a,b]$. Ukoliko je $\mcl{P}_2\psj \mcl{P}_1$, onda kazhemo da je $\mcl{P}_1$ finija od $\mcl{P}_2$. Podela 
    $\mcl{P}_1\cup \mcl{P}_2$ je superpozicija podela $\mcl{P}_1$ i $\mcl{P}_2$ (i ona je, trivijalno, finija od obe podele).

    Ako su $\mcl{I}_1,\mcl{I}_2,\dotso,\mcl{I}_n\psj \R$ intervali, podela kvadra $\mcl{J}=\mcl{I}_1 \times \mcl{I}_2 \times \dotso \times \mcl{I}_n\psj \R^n$ je
    $\mcl{P}=(\mcl{P}_1,\mcl{P}_2,\dotso, \mcl{P}_n)$, gde je $\mcl{P}_j$ podela intervala $\mcl{I}_j$, za svako $j\in \{1,2,\dotso,n\}$.

    Analogno definishemo pojmove finije podele i superpozicije podela kvadra. Kazhemo da je jedna podela kvadra finija od druge podele kvadra, ako je \zn finija po svakoj ko-ordinati\zng. 
    Superpozicija podela kvadra je podela dobijena superpozicijama \zn po svim koordinatama\zng. Podeoni kvadri su proizvodi podeonih intervala.

    \begin{de}
    	Podela kvadra $\mcl{J}$ sa uochenim tachkama je par $(\mcl{P},c)$, gde je $\mcl{P}$ podela kvadra $\mcl{J}$ sa podeonim kvadrima $\mcl{J}_1,\dotso\mcl{J}_l$,
	a $c=(c_1,c_2,\dotso,c_l)$ $l$-torka brojeva takva da je $(\forall i)c_i\in\mcl{J}_i$ ($c_i$ je uochena tachka kvadra $\mcl{J}_i$).
    \end{de}

    Sa $\mcl{P}_\mcl{J}$ oznachavamo skup svih podela kvadra $\mcl{J}$, a sa $\widetilde{\mcl{P}}_\mcl{J}$ skup svih podela kvadra $\mcl{J}$ sa uochenim tachkama.
    Identifikujemo $\mcl{P}\in \mcl{P}_\mcl{J}$ sa $(\mcl{P},\emptyset)\in \widetilde{\mcl{P}}_\mcl{J}$ i smatramo da je $\mcl{P}_\mcl{J}\psj\widetilde{\mcl{P}}_\mcl{J}$
    
    \begin{de}
    	Parametar podele $\mcl{P}=(\mcl{P}_1,\mcl{P}_2,\dotso, \mcl{P}_n)$ (u oznaci $\lambda(P)$) je maksimum duzhina svih podeonih intervala podela $\mcl{P}_1,\mcl{P}_2,\dotso, \mcl{P}_n$.
    \end{de}

    Neka je, za $\delta>0$:
    \[B_\delta := \{(\mcl{P},c)\in \widetilde{\mcl{P}}_\mcl{J}\mid \lambda(P)<\delta\}.\]
    Dokazhimo da je tada
    \[\b_\mcl{J}:=\{B_\delta \mid \delta > 0\}\]
    baza filtera na $\widetilde{\mcl{P}}_\mcl{J}$. 

\end{document}
