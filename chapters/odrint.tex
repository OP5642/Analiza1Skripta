\documentclass[../main.tex]{subfiles}

\usepackage[OT2]{fontenc}
\usepackage[english,serbianc]{babel} 
\usepackage{listings}%dozvoljava nam da dodajemo delove koda
\usepackage{amsmath}%dozvoljava nam neke stvari iz matematike
\usepackage{amsthm}%theorem stvari
\usepackage{dsfont}%npr. za boldovanu jedinicu
\usepackage{amsfonts}%dozvoljava upotrebu raznih fontova
\usepackage{mathrsfs}%za neke fontove, npr mathscr za filtere
\usepackage{mathtools} %svashta, npr. za pisanje teksta iznad \iff
\usepackage{amssymb}%dozvoljava upotrebu raznih simbola puput measuredangle
\usepackage{color} %koristimo boje
\usepackage{enumitem,framed} %za leftmargin u itemiz
\usepackage{indentfirst}
\usepackage[lmargin=2.0cm, rmargin=2.0cm,tmargin=2.50cm,bmargin=2.50cm]{geometry}

\usepackage{subfiles} %najbolje ovo na kraju koristiti

\definecolor{mygreen}{rgb}{0,0.5,0}
\definecolor{mygray}{rgb}{0.7,0.7,0.7}
\definecolor{mymauve}{rgb}{0.58,0,0.82}
 
\righthyphenmin 2 


%Podesavanje izgleda koda
\lstset{ 
 backgroundcolor=\color{mygray}, % choose the background color
 basicstyle=\footnotesize, % the size of the fonts that are used for the code
 breakatwhitespace=false, % sets if automatic breaks should only happen at whitespace
 breaklines=true, % sets automatic line breaking
 captionpos=b, % sets the caption-position to bottom
 commentstyle=\color{blue}, % comment style
 deletekeywords={...}, % if you want to delete keywords from the given language
 escapeinside={\%*}{*)}, % if you want to add LaTeX within your code
 extendedchars=true, % lets you use non-ASCII characters; for 8-bits encodings only, does not work with UTF-8
 frame=single, % adds a frame around the code
 keepspaces=true, % keeps spaces in text, useful for keeping indentation of code (possibly needs columns=flexible)
 keywordstyle=\color{mygreen}, % keyword style
 language=TeX, %defining the prefered language, can be changned in []
 morekeywords={*,...}, % if you want to add more keywords to the set
 numbers=left, % where to put the line-numbers; possible values are (none, left, right)
 numbersep=5pt, % how far the line-numbers are from the cod
 numberstyle=\tiny\color{mygray}, % the style that is used for the line-numbers
 rulecolor=\color{black}, % if not set, the frame-color may be changed on line-breaks within not-black text (e.g. comments (green here))
 showspaces=false, % show spaces everywhere adding particular underscores; it overrides 'showstringspaces'
 showstringspaces=false, % underline spaces within strings only
 showtabs=false, % show tabs within strings adding particular underscores
 stepnumber=1, % the step between two line-numbers. If it's 1, each line will be numbered
 stringstyle=\color{mymauve}, % string literal style
 tabsize=2, % sets default tabsize to 2 spaces
 title=\lstname % show the filename of files included with \lstinputlisting; also try caption instead of title
}
%Kraj podesavanja izgleda koda
\begin{document}

\section{Rimanov integral}

    Cilj je definisati (eventualno i izrachunati) povrshinu ispod grafika zadate funk-cije $f:[a,b]\to \R$ (bitno je da je funkcija na intervalu). 
    Ovo je problem koji su i Stari Grci, Arhimed i Eudoks, umeli da reshe za prilichno
    shiroke klase funkcija.

    Moguc1e je pristpupiti problemu sa vishe gledishta, ili preko posmatranja limesa, ili preko supremuma i infimuma. Prvi nachin je Rimanov, a drugi Darbuov i oba ima-ju
    svoje mane, kao i prednosti. Kada dokazhemo njihovu ekvivalenciju, moc1i c1emo da upotrebljavamo oba u zavisnosti od toga koji nam vishe odgovara u datoj situaciji.

\subsection{Veza sa primitivnom funkcijom}

    Pretpostavimo da je funkcija $f$ neprekidna na intervalu $[a,b]$ i oznachimo sa $P(x)$ povrshinu ispod grafika funkcije $f$ na intervalu $[a,x]$.
    Dakle, ako imamo $h \to 0$, onda je i $f(x+h)\to f(x)$, odakle sledi $P(x+h)-P(x) \sim f(x) h$ (tj. razliku povrshina posmatramo zapravo kao povrshinu pravougaonika).
    Odavde imamo da je \[f(x) \sim \dfrac{P(x+h)-P(x)}{h}\xrightarrow{h\to 0}P'(x)\\
	\implies P'(x)=f(x).
    \]
    Dakle, ako je $f\cont [a,b]$, $P$ je primitivna za $f$.

    Naglasimo da je ovo neformalan dokaz, sluzhi nam chisto radi intuicije. Ovo tvrdjenje c1emo kasnije nazivati osnovnom teoremom integralnog rachuna.
    Ovo nam daje intuitivnu predstavu o sledec1oj definiciji.

    \begin{de}
    	Neka je $F:[a,b]\to \C$ primitivna funkcija neprekidne funkcije $f:[a,b]\to \C$. Njutnov integral funkcije $f$ je 
	\[ \int_a^bf(x)dx=F(b)-F(a).\]
    \end{de}
    Iz jedinstvenosti primtivne funkcije $F$ za $f$ na intervalu (do na aditivnu konstan-tu) sledi da je definicija dobra, tj. da Njutnov integral ne zavisi od odabira
    primitivne funkcije.

    Razlika $F(b)-F(a)$ se oznachava i sa $F|_a^b$, $F(x)|_a^b$, $F(x)|_{x=a}^{x=b}$, a umesto $\int_a^bf(x)dx=F(x)|_a^b$ pishemo $\int_a^bf=F|_b^a$.
    
    No, prirodno je postaviti pitanje kada uopshte postoji primitvna funkcija? Odavde je jasno da za neprekidne postoji primitivna funkcija, ali to je veoma
    jaka pretpostvka. Iz datih definicija i primera ne mozhemo da zakljuchimo kada postoji primitivna funkcija date funkcije. Odogovor c1e nam dati formalna definicija
    Rimanovog integrala.

    Mozhemo formulisati i opsthiji problem. Naime, zashto se zausaviti na povrshini? Chesto c1e nas zanimati i zapremina pa chak i vishe dimenziona zapremina. No pre toga
    treba definisati pojam analogan intervalu u vishe dimenzija.

    \begin{de}
    	Neka su $\mcl{I}_1,\mcl{I}_2,\dotso,\mcl{I}_n$ intervali u $\R$. Dekartov proizvod $\mcl{J}=\mcl{I}_1 \times \mcl{I}_2 \times \dotso \times \mcl{I}_n$ nazivamo 
	$n$-dimenzionim	kvadrom.
    \end{de}

    Krajnji cilj je definisati (ukoliko je moguc1e i izrachunati) $(n+1)$-dimenzionu zapreminu ispod grafika funkcije $f:\mcl{J} \to \R$.

    U sluchaju $n=1$ imamo \zn 2-dimenzionu zapreminu\zng, shto je zapravo povrshina, a u sluchaju $n=2$ imamo \zn 3-dimenzionu zapreminu\zng, shto je zapravo 
    standardna zapremina.

\subsection{Kostrukcija Rimanovog integrala}

    \begin{de}
    	Podela intervala $[a,b]$ je konachan skup tachaka $\mcl{P}=\{x_0,x_1,\dotso,x_k\}$, takvih da je $a=x_0<x_1<x_2<\dotso<x_k=b$. Svaku tachku koja je chlan podele
	intervala nazivamo podeonom tachkom, a svaki od intervala $[x_i,x_{i+1}],i\in\{0,1,\dotso,n-1\}$ nazivamo podeonim intervalom.
    \end{de}

    Neka su $\mcl{P}_1$ i $\mcl{P}_2$ podele intervala $[a,b]$. Ukoliko je $\mcl{P}_2\psj \mcl{P}_1$, onda kazhemo da je $\mcl{P}_1$ finija od $\mcl{P}_2$. Podela 
    $\mcl{P}_1\cup \mcl{P}_2$ je superpozicija podela $\mcl{P}_1$ i $\mcl{P}_2$ (i ona je, trivijalno, finija od obe podele).

    Ako su $\mcl{I}_1,\mcl{I}_2,\dotso,\mcl{I}_n\psj \R$ intervali, podela kvadra $\mcl{J}=\mcl{I}_1 \times \mcl{I}_2 \times \dotso \times \mcl{I}_n\psj \R^n$ je
    $\mcl{P}=(\mcl{P}_1,\mcl{P}_2,\dotso, \mcl{P}_n)$, gde je $\mcl{P}_j$ podela intervala $\mcl{I}_j$, za svako $j\in \{1,2,\dotso,n\}$.

    Analogno definishemo pojmove finije podele i superpozicije podela kvadra. Kazhemo da je jedna podela kvadra finija od druge podele kvadra, ako je \zn finija po svakoj ko-ordinati\zng. 
    Superpozicija podela kvadra je podela dobijena superpozicijama \zn po svim koordinatama\zng. Podeoni kvadri su proizvodi podeonih intervala.

    \begin{de}
    	Podela kvadra $\mcl{J}$ sa uochenim tachkama je par $(\mcl{P},c)$, gde je $\mcl{P}$ podela kvadra $\mcl{J}$ sa podeonim kvadrima $\mcl{J}_1,\dotso\mcl{J}_l$,
	a $c=(c_1,c_2,\dotso,c_l)$ $l$-torka brojeva takva da je $(\forall i)c_i\in\mcl{J}_i$ ($c_i$ je uochena tachka kvadra $\mcl{J}_i$).
    \end{de}

    Sa $\mcl{P}_\mcl{J}$ oznachavamo skup svih podela kvadra $\mcl{J}$, a sa $\widetilde{\mcl{P}}_\mcl{J}$ skup svih podela kvadra $\mcl{J}$ sa uochenim tachkama.
    Identifikujemo $\mcl{P}\in \mcl{P}_\mcl{J}$ sa $(\mcl{P},\emptyset)\in \widetilde{\mcl{P}}_\mcl{J}$ i smatramo da je $\mcl{P}_\mcl{J}\psj\widetilde{\mcl{P}}_\mcl{J}$
    
    \begin{de}
    	Parametar podele $\mcl{P}=(\mcl{P}_1,\mcl{P}_2,\dotso, \mcl{P}_n)$ (u oznaci $\lambda(\mcl{P})$) je maksimum duzhina svih podeonih intervala podela $\mcl{P}_1,\mcl{P}_2,\dotso, \mcl{P}_n$.
    \end{de}

    Neka je, za $\delta>0$:
    \[B_\delta := \{(\mcl{P},c)\in \widetilde{\mcl{P}}_\mcl{J}\mid \lambda(P)<\delta\}.\]
    Dokazhimo da je tada
    \[\b_\mcl{J}:=\{B_\delta \mid \delta > 0\}\]
    baza filtera na $\widetilde{\mcl{P}}_\mcl{J}$. 

    Najpre, primetimo da je uopshte smisleno govoriti o filteru u ovoj situaciji (tj. o bazi filtera). Ochigledno, svaki $B_\delta$ je podskup od $\widetilde{\mcl{P}}_\J$,
    pa $\b_\J$ zaista jeste familija podskupova $\widetilde{\mcl{P}}_\J$. Najpre, primetimo da za dovoljno veliko $\delta$ vazhi $B_\delta=\widetilde{\mcl{P}}_\J$, jer je 
    kvadar $\J$ konachan skup (mozhemo uzeti da $\delta$ bude najduzha njegova ivica uvec1ana za 1, zbog uslova da je paramtar podele strogo manji od $\delta$, ovo ima smisla
    jer prichamo o ogranichenim kvadrima). Takodje, evidentno je da je moguc1e izabrati \zn dovoljno finu podelu kvadra $\J$\zng, pa je nemoguc1e da je bilo koji od
    $B_\delta = \emptyset$, pa $\emptyset \notin \b_\J$. Dalje, familija $\b_\J$ jeste zatvorena u odnosu na presek jer je 
    \[B_{\delta_1}\cap B_{\delta_2}=B_{\min\{\delta_1,\delta_2\}}.\]
    Dakle, $\b_\J$  zaista jeste baza filtera na $\widetilde{\mcl{P}}_\mcl{J}$. Filter njome generisan oznachavamo sa $\f_\J$ i on c1e biti od kljuchnog znachaja za 
    definisanje integrala.

    \begin{pr}
    	\[X \in \f_\J \iff (\exists \delta > 0) (\lambda(\mcl{P})<\delta \implies (\mcl{P},c)\in X)\]
    \end{pr}

    \begin{proof}
    	Kao shto smo ranije videli, filter generishemo od baze uzimanjem josh svih nadsku-pova chlanova baze. U ovom sluchaju chlanovi baze su skupovi svih podela
	chiji je paramtar manji od nekog $\delta>0$. Dakle, svaki chlan filtera c1e biti neki skup podela, koji sadrzhi barem one podele kojima je paramtar
	manji od nekog $\delta>0$, a sadrzhi i josh neke podele, koje nama nisu od znachaja. Prethodna rechenica je neformalni zapis trazhenog tvrdjenja.
    \end{proof}

    Filter $\f_\J[\mcl{P}_\J]$ na $\mcl{P}_\J$ indukovan filterom $\f_\J$ na $\widetilde{\mcl{P}}_\J$ oznachavamo krac1e istom oznakom $\f_\J$. Ovo ima smisla da radimo jer
    smo vec1 naglasili da identifikujemo podele bez uochenih tachaka kao podele sa praznim skupom uochenih tachaka.

    \begin{de}
    	Zapremina kvadra $\J=[a_1,b_1]\times \dotso [a_n,b_n]$ je \[\mu(\J):=\prod_{j=1}^n(b_j-a_j).\]
    \end{de}

    Specijalno, za $n=1$ velichina $\mu(\J)$ predstavlja duzhinu intervala.

    \begin{de}
    	Unutrashnjost kvadra $\J = [a_1,b_1]\times \dotso [a_n,b_n]$ je otvoreni kvadar $$\overset{\circ}{\J} = ]a_1,b_1[\times \dotso \times ]a_n,b_n[.$$
    \end{de}

    \begin{tvr}
    	Neka su $\J$ i $\J_1,\dotso,\J_k$ kvdri u $\R^n$. Tada vazhi: 
	\begin{itemize}
	    \item[(1)] $\mu(p\J)=|p|^n\mu(J)$, za $p \in \R$. 
	    \item[(2)] $\J \psj \cup_{i=1}^k \implies \mu(\J) \mj \sum_{i=1}^k \mu(\J_i)$
	    \item[(3)] $\J=\cup_{i=1}^k \land ( p \neq q \implies \overset{\circ}{\J_p} \cap \overset{\circ}{\J_q} = \emptyset) \implies \mu(\J) = \sum_{i=1}^k \mu(\J_i)$
	\end{itemize}
    \end{tvr}

    \begin{proof}
    	Sledi direktno iz definicije zapremine kvadra.
    \end{proof}

    \begin{de}
    	neka je $\J \psj \R^n$ $n$-dimenzioni kvadar, $(\mcl{P},c)$ podela kvadra $\J$ sa uochenim tachkama $ c = \{c_1,\dotso, c_l\}$ i podeonim kvadrima $\J_1,\dotso,\J_l$
	i $f:\J \to \C$ funkcija. Velichina 
	\[ \ris f P c:= \sum_{i=1}^{l}f(c_i)\mu(\J_i)\]
	naziva se Rimanovom integralnom sumom. Limes (ako postoji) 
	\begin{align}
	    \label{defrimint}
	    \int_\J f(x)dx:=\lim_{\f_\J} \ris f P c 
	\end{align}
	funkcije 
	\[\ris f \cdot \cdot : \widetilde{\mcl{P}}_\J \to \C, (\mcl{P},c) \mapsto \ris f P c\]
	naziva se Rimanovim integralom funkcije $f$.
    \end{de}

    Chesto koristimo i druge oznake za  $\int_\J f(x)dx$:
    \[\int_\J f, \int_\J f(x_1,x_2,\dotso, x_n)dx_1\dotso dx_n.\]
    SPecijalno, za $n=1$ i $\J=[a,b]$ pishemo 
    \[\int_{[a,b]}f(x)dx = \int_a^b f(x)dx.\]
    Ako limes u \ref{defrimint} postoji (i konachan je), kazhemo da je funkcija $f$ integrabilna po Rimanu na kvadru $\J$. Skup svih funkcija za koje postoji limes
    \ref{defrimint} na kvadru $\J$ obelezhavamo sa $\mcl{R}(\J)$ (to je skup svih funkcija integrabilnih po Rimanu na posmatranom kvadru).

    \begin{tvr}
    	Neophodan uslov integrabilnosti po Rimanu jeste da je funkcija ograni-chena na posmatranom kvadru. Tj.
	\[\mcl{R}(\J) \psj B(\J)\]
    \end{tvr}

    \begin{proof}
    	Neka je $\mcl{P}$ proizvoljna podela kvadra $\J$. Pretpostavimo suprotno, tj. da funkcija nije ogranichana na $\J$. Ako funkcija nije ogranichena na 
	$\J$, to znachi da postoji neki kvadar podele $\J_i$ na kojem funkijca nije ogranichena (u suprotnom bi bila ogrnichena). Uzmimo dve podele, $(\mcl{P},c')$ i 
	$(\mcl{P},c'')$, koje se razlikuju jedino po odabiru uochene tachke $c'_i \neq c''_i$ na kvadru $\J_i$. Tada je 
	\[|\ris f P c' - \ris f P c''| = |f(c'_i)-f(c''_i)|\mu(\J_i),\]
	shto je proizvoljno veliko (jer je $|f(c'_i)-f(c''_i)|$ proizvoljno veliko), pa funkcija $\ris f \cdot \cdot$ ne ispunjava Koshijev uslov.
    \end{proof}

    \nap $$f \in \mcl{R} \implies \lim_{\f_\J}\ris f P c$$ ne zavisi od $c$.

    \begin{tvr}
	Funkcija $f: \J \to \C$ je integrablna po Rimanu akko su integrabilne funkcije $\Re f, \Im f: \J \to \R$. Ako je $f$ integrabilna, onda vazhi:
	\[\int_\J f(x)dx = \int_\J \Re f(x) dx + i\int_\J\Im f(x) dx.\]
    \end{tvr}
    \begin{proof}
	\begin{align*}
	    \ris{f}{P}{c} = &\ris{\Re f+ i\Im f}{P}{c}\\
	    = &\ris{\Re f}{P}{c} + i \ris{\Im f}{P}{c}\\
	    \implies &\Re \ris{f}{P}{c} = \ris{\Re f}{P}{c}\\ 
	    & \Im \ris{f}{P}{c} = \ris{\Im f}{P}{c}\\
	    \implies &\text{ Dokaz je direktna posledica odgovarajuc1eg tvrdjenja za } \lim_\f.
	\end{align*}     \end{proof}


    \begin{posl}
    	Integrabilnost i integrali kompleksnih funkcija svode se na inte-grabilnost i integral realnih.
    \end{posl}

\section{Darbuov integral}

    Dosada smo se bavili Rimanovim integralom. Nashe sledec1e interesovanje bic1e Darbu-ov integral, za koji c1emo kasnije dokazati da je ekvivalentan Rimanovom,
    ali postojanje oba nam olakshava dokaze i omoguc1ava nam razlichite pristupe istom problemu.

\subsection{Gornja i donja Darbuova suma}
    
    Neka je $\J\psj \R^n$ kvadar u $n$-dimenzija i $\mcl{P}$ neka njegova podela i $\J_1, \dotso, \J_l$ kvadri podele $\mcl{P}$ i $f:\J \to \R$ ogranich1ena realna funkcija.
    Oznachmo sa $m_i$ i $M_i$  infimum i supremum funkcije $f$ na kvadru podele $\J_i$, redom.

    \begin{de}
    	Velichine 
	\[s(f,\mcl{P})=\sum_{i=1}^lm_i\mu(\J_i)\text{ i }S(f,\mcl{P})=\sum_{i=1}^lM_i\mu(\J_i)\]
	nazivaju se donja i gornja Darbuova suma funkcije $f$ na kvadru $\J$ u odnosu na podelu $\mcl{P}$. 
    \end{de}
    

    \begin{lem}
	\begin{enumerate}
	    \item $\ds{\dds f P = \inf_c \ris f P c \mj \sup_c \ris f P c = \gds f P}$
	    \item Ako je $\mcl{P}_1$ podela finija od podele $\mcl{P}_2$, onda je 
	    	\[s(f,\mcl{P}_2) \leqslant s(f,\mcl{P}_1) \leqslant S(f,\mcl{P}_1) \leqslant S(f,\mcl{P}_2).\] 
	    \item Za proizvoljne podele $\mcl{P}_1$ i $\mcl{P}_2$ je $s(f,\mcl{P}_1) \mj S(f,\mcl{P}_2)$. 
	\end{enumerate}
    \end{lem}

    \begin{proof}
    	Dokaz prve stavke sledi direktno iz definicije gornje i donje Darbuove sume. Druga stavka iziskuje neshto vishe posla, ali je opet intuitivna.
	Svaki kvadar grublje podele se mozhe zapisati kao konachna unija podeonih kvadara finije podele. Dakle, za svaki kvadar $\J_i$ podele $\mcl{P}_2$je 
	\[\J_i = \cup_{j=1}^k \J_i^j.\] Kako je svaki $\J_i^j$ podskup kvadra $\J_i$, onda  je $\inf f(\J_i) \mj \inf f(\J_i^j)$.
	\begin{align*}
	    \inf f(\J_i)\mu(\J_i) &= \inf f(\ds{\cup_{j=1}^k \J_i^j})\sum_{j=1}^k\mu(\J_i^j)\\
	    &\mj \sum_{j=1}^k \inf f(\J_i^j)\mu(\J_i^j). 
	\end{align*}
	Sumiranjem po svim podeonim kvadrima dobijamo prvu nejednakost u 2. Slichnim postup-kom (ili primenom dobijenog dela na $-f$ umesto na $f$) dobijamo i trec1u nejednakost.
	Druga nejednakost je direktna posledica svojstva supremuma i infumuma (infimum je manji od supremuma, pa opet sumiranjem po svim podeonim kvadrima dobijamo trazhenu nejednakost).
	Svojstvo 3 sledi iz svojstva 2 posmatranjem superpozicije podela.
	\[s(f,\mcl{P}_1) \mj s(f,\mcl{P}_1 \cup \mcl{P}_2)\mj S(f,\mcl{P}_1\cup \mcl{P}_2)\mj S(f,\mcl{P}_2)\]
    \end{proof}


\end{document}
